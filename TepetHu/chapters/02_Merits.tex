\mychapter{merits}{Merits}


\begin{Merit}{The Amulet of Delepitorae}{%
    Type=Artifact,
    Rating=2,
    Reference=\cite*[p.~159]{ex3}
}{amulet-of-delepitorae}
    An orichalcum hearthstone amulet found in \nameref{merit:cirrus-sanctum}.
    The body of the amulet is eye-shaped, carved with intricate Old Realm
    caligraphy. The socket contains several rings attached to the body. When a
    hearthstone is placed near the amulet, the rings spin to create an opening,
    and then spin back to secure the hearthstone within the frame.

    This amulet was specifically designed to hold
    \nameref{hearthstone:eye-of-delepitorae}. The amulet, the manse, and the
    hearthstone were all designed in the First Age by a Twilight
    sorcerer-engineer. Whether he or she was Delepitorae or if the name came
    later is unknown. While the amulet is capable of socketing any hearthstone,
    The Eye fits perfectly in its rings.

    \Artifact{%
        Attunement=1m,
        Type={Hearthstone Amulet},
        HearthstoneSockets=1,
        Reference=\cite*[p~601]{ex3}
    }
\end{Merit}


\begin{Merit}{Chain Lightning}{%
    Description=Blue Jade Infinite Chakram,
    Type=Artifact,
    Rating=3,
    Reference=\cite*[p.~159]{ex3}
}{chain-lightning}
    A circular blade of blue jade inlaid with starmetal filigree. Even when
    held, Chain Lightning tingles with power. When thrown, the blade crackles
    to life with a shower of sparks. Bolts of electrical essence arc to any
    metal objects near enough its path. With each collision and ricochet, Chain
    Lightning emits a sharp crack and burst of electricity.

    \Artifact{%
        Attunement=5m,
        Type={Light (+10 Damage, 3 Overwhelming)},
        Tags={Lethal, Thrown (Medium), Cutting, Special},
        Reference=\cite*[p~597]{ex3}
    }
    \RangedWeapon[merit:chain-lightning]{%
        Ability=\Stat{Thrown} + 1,
        Damage=10,
        Overwhelming=5,
        Keywords={Lethal, Thrown (Medium), Cutting, Special},
    }{%
        Close=5,
        Short=4,
        Medium=3,
        Long=0,
        Extreme=-2
    }

    \CharmSubSection{chain-lightning-evocations}{Evocations of Chain Lightning}

    Once the Exalt has attuned to Chain Lightning, \tldr{when he is in Air Aura,
    Chain Lightning provides \Modifier{+1} to Accuracy. In strongly
    air-aspected environments, such as high winds or thunderstorms, it adds
    \Pool{\StatVal{Essence}} to Accuracy instead.}

    \begin{Charm}{Shocking Strike}{%
        Cost=2m,
        Duration=Instant,
        Keywords={Air, Uniform},
        Mins={Athletics 1, Essence 1},
        Type=Supplemental
    }{shocking-strike}
        Armor may deflect a blade, but it also serves as a channel for
        electrical essence. \tldr{An attack made with Chain Lightning ignores
        \Pool{\StatVal{Essence}} soak or hardness from metal armor, including
        the five magical metals.}
    \end{Charm}
\end{Merit}


\begin{Merit}{The Cirrus Sanctum}{%
    Description=Greater Manse,
    Type=Manse,
    Rating=5,
    Reference=\cite*[p.~163]{ex3}
}{cirrus-sanctum}
    On top of a permanent storm cloud, wandering \emph{somewhere} in the seas
    to the Southwest of the Blessed Isle, sits a beautiful tower fortress: The
    Cirrus Sanctum. The tower is not visible from sea level; one must fly up
    to the cloud level to see its white stone gleaming in the sun.

    \begin{figure}[h]
        \includegraphics[width=\linewidth]{CirrusSanctum.jpg}
        \caption*{Basic idea, but missing the cloud}
    \end{figure}

    The Sanctum was built in the First Age by a Twilight sorcerer-engineer.
    That history has been lost to time for the most part. Since the Usurpation,
    the Sanctum has had several masters, eventually passing into the ownership
    of House Tepet. Tepet Hu has been granted use of the manse by his House to
    further his arcane research.

    The tower consists of several floors. The "ground" floor --- level with the
    top of the cloud --- is an a entrance hall. The next floor up is a main
    living area. It has a relatively modest dining room and sitting room. Up
    from there are bedrooms. Up still is the library. And finally, the top
    floor is the hearthstone room, which also serves as a sorcerous workshop,
    including an orichalcum binding circle inlaid into the floor. There is also
    a basement with a kitchen and servant quarters.

    Together the Sanctum and its cloud drift along the leylines of the
    Sourthwestern seas, slowly drawn by the currents of essence in the air.
    Predicting the location of the manse is something like predicting the
    weather. It's easy enough to know where it will be tomorrow, but much
    harder a month from now. It is an \Pool{\StatVal{Intelligence} +
    \StatVal{Occult}} roll to predict its movements, starting at difficulty 2
    but increasing by 1 for each month, to a max of 5. Being attuned to the
    Sanctum caps the difficulty at 3; the Exalt gains an intuitive
    understanding of the manse's movements, but still must understand the
    currents of essence.


    \begin{Hearthstone}{The Eye of Delepitorae}{%
        Type=Greater Air Hearthstone,
        Keywords=Linked
    }{eye-of-delepitorae}
        \begin{figure}[b]
            \centering
            \includegraphics[height=1.5in]{Hearthstone.png}
            \caption*{\itshape The Eye of Delepitorae}
        \end{figure}
        This hearthstone is an azure blue orb, perfectly spherical. Its depths
        are a dark black cloud. The Eye of Delepitorae holds a pool of up to 10
        sorcerous motes. As long as it is socketed into an attuned artifact,
        it allows the bearer to draw up to \Pool{\StatVal{Essence}} sorcerous
        motes from its pool per spell. Until the spell is cast or lost, the
        stone glows brightly.

        The Eye of Depepitorae recharges one sorcerous mote per day, at twilight,
        or two within its home manse. The size of the dark cloud in the center
        of the Eye varies inversely with its charge.

        Once per story, the wearer may draw 10 sorcerous motes from The Eye of
        Delepitorae, regardless of it current pool. Once the spell is either cast
        or lost, The Eye's sorcerous mote pool empties and the stone turns jet
        black. It will not recharge sorcerous motes unless left to charge in
        its home manse for one month, coming to life fully charged at twilight
        on the 30th day.
    \end{Hearthstone}
\end{Merit}


\begin{Merit}{Language: Dragontongue}{%
    Type=Language,
    Rating=1,
    Reference=\cite*[p.~162]{ex3}
}{language-dragontongue}
    Derived from recovered elements of a priestly language that was lost during
    the shogunate, luminaries of House Mnemon birthed and spread this language
    over the course of two centuries. It is a mix of Old Realm and High Realm,
    with elements of its lost shogunate tongue, and excludes the mind from the
    wider, more dangerous concepts inherent to Old Realm, keeping a person in
    mind of the Dragons, the Poles, the natural world and the Perfected
    Hierarchy. It is a beautiful language more than a scholarly one, and even
    in satrapies which have been thoroughly suppressed by the Realm, there is a
    rush by savants and poets to learn this language of poets and princes. Its
    written form utilizes very challenging yet beautiful brushstrokes.
\end{Merit}


\begin{Merit}{Language: Old Realm}{%
    Type=Language,
    Rating=1,
    Reference=\cite*[p.~162]{ex3}
}{language-old-realm}
    The native language of the spirits and those that created them, as well as
    of the Fair Folk. It was widely spoken in the First Age, especially by the
    savants and sorcerer-engineers, and used for many official documents.
    Characters must have Lore 1+ or Occult 1+ to learn this language. There
    exist several styles for writing Old Realm, the most extravagant of which
    is a complex glyphic system where symbol arrangement is as important as
    symbol choice, and the same phrase might be read in several ways, often as
    a deliberate choice by the writer intended to impart subtle and layered
    meaning.
\end{Merit}


\DocumentColumnBreak
\begin{Merit}{House Stipend}{%
    Type=Resources,
    Rating=3,
    Reference=\cite*[p.~164]{ex3}
}{house-stipend}
    \TBW

    \mysubsection{breastplate}{Feathersteel Breastplate}

    One item purchased with his resources is a breastplate made of
    feathersteel, a rare, light metal mined in the ice-covered mountains of
    the far North. All armor made of feathersteel is somewhat ligther than
    normal and never rusts.

    \Armor[subsection:breastplate]{%
        Soak=3,
        Keywords=Silent
    }
\end{Merit}


\begin{Merit}{Influence (House Tepet)}{%
    Type=Influence,
    Rating=2,
    Reference=\cite*[p.~162]{ex3}
}{influence}
    \TBW
\end{Merit}


\begin{Merit}{Professorship at the Heptagram}{%
    Type=Backing,
    Rating=3,
    Reference=\cite*[p.~159]{ex3}
}{professorship}
    \TBW
\end{Merit}


\printbibliography[title=References]

